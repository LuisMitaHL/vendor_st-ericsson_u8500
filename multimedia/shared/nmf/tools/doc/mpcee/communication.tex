\chapter{Communications}
In this chapter we will describe intra MMDSP and inter MMDSP communications.
In both case it will result in the insertion of an element in the
scheduler fifo by a call to Sched\_Event() function.

Communication is handle by NMF generate code. NMF toolset will generate so
called stub code which wrapped client side communication and skeleton code which
wrapped server side communication.


\section{Inter MMDSP communications}
\subsection{Low level primitive}
\subsubsection{Fifo}
Fifo are the low level object use to insure NMF inter communications. These fifo
are deployed by CM driver code and consist of a physically continuous memory
which is shared by both side. Fifo consist of two part, a header that contain
metadata need by communications primitive and a payload of an unknow content.
Payload consist of n elements of the same size.

Header is describe by the t\_nmf\_fifo\_desc c structure (Listing
\ref{tnmffifodesc})
\begin{lstlisting}[caption=fifo header structure, label=tnmffifodesc]
typedef struct {
    t_semaphore_id semId;   // semaphore index to use
    t_uint16 elemSize;      // Size in 16 bits size of a fifo elem
    t_uint16 fifoFullValue; // nb of elem request by user
    t_uint16 readIndex;     // Current read index
    t_uint16 writeIndex;    // Current write index
    t_uint16 wrappingValue; // nb of elem plus one request by user
    t_uint32 extendedField; // pointer to memory for which role vary
                            // according to context
    } t_nmf_fifo_desc;
\end{lstlisting}

Payload then follow header and are composed of n+1 elements (we allocate one
more element to distinguish fifo empty case from fifo full case) of elemSize*2
bytes.

Fifo object offer following methods for manipulation :
\begin{itemize}
\item fifo\_getAndAckNextElemToWritePointer() : get next fifo element for
write access and advance writeIndex pointer.
\item fifo\_getAndAckNextElemToReadPointer() : get next fifo element for
read access and advance readIndex pointer.
\item fifo\_getNextElemToWritePointer() : get next fifo element for
write access without advance of the writeIndex pointer.
\item fifo\_getNextElemToReadPointer() : get next fifo element for
read access without advance of the readIndex pointer.
\item fifo\_acknowledgeRead() : increment readIndex pointer. This has for effect
to free an element which can then be use for write.
\item fifo\_acknowledgeWrite() : increment writeIndex pointer. This has for
effect to add and element which can be use for read.
\item fifo\_coms\_acknowledgeWriteAndInterruptGeneration() : same as
fifo\_acknowledgeWrite() but also generate an interrupt.
\end{itemize}

\subsubsection{Signaling a new message}
To signal a receiver we use HSEM hardware ip to generate an interruption. HSEM
ip had 32 semaphores and each of them can generate an irq toward arm and MMDSP
cores.
We use a unique semaphore for each communication channel (coreA to CoreB) which
allow reveiver to know which core want to communicate and so in which com fifo
it had to look for new message.

\subsection{High level}
To insure communication between coreA and coreB there is the instantiations of
two called coms fifo, one insure communication between coreA and coreB and the
other one between coreB and coreA. These fifos are deployed by the CM driver
when a MMDSP is power on. Number of element in these fifo have a fixed size set
in CM driver code. Carried payload consist of information needed to retriewe
which interface, method and parameters need to be executed.

Then we have the so called params fifo. These fifo are deployed by the CM driver
each time we bind an interface between host and MMDSP or between two MMDSP.
There goal is to carry fonctions parameters. Number of element in these fifo are
given by the user when he request a binding.

To finish CM driver also deployed a stack of elements that will be push into the
scheduler to execute the remote function call.
\subsubsection{Coms fifo}
Coms fifo are fifo of ARM\_DSP\_EVENT\_FIFO\_SIZE (128 is the current value)
without extendedSharedFields.

Payload carry by a com fifo can be describe by the following c structure:
\begin{lstlisting}[caption=com fifo payload description, label=comPayload]
typedef struct {
    t_uint32 methodIndex; // index of the method to execute
    t_uint32 paramIndex; // pointer to the param fifo elem that carry parameters
    t_uint32 extFieldPointer; // pointer to the extented field of interface
} com_fifo_payload;
\end{lstlisting}

\subsubsection{Params fifo}
Param fifo size is the number of element request by user during interface
binding.

For MMDSP to MMDSP or host to MMDSP communications ExtendedSharedFields is used
to :
\begin{itemize}
  \item to store address of TOP attribute which is the head of event link list.
  \item to store address of methods provide by the interface.
\end{itemize}
So in this case ExtendedSharedFields has NumberOfMethodsProvideByInterface+1
elements.

For MMDSP to host communication ExtendedSharedFields has two elements :
\begin{itemize}
  \item a pointer to an opaque context structure.
  \item size in bytes of a fifo element.
\end{itemize}

Payload carry marshalled function parameters. Size of an element size is given
by the max of the parameters size (in bytes) of all the method of an interface.

\subsubsection{Event link list}
For Host to MMPDSP or MMDSP to MMDSP communications CM driver create an event
link list of element that will be use to be push into scheduler. It also had the
same number of elements that the corresponding param fifo. The link list head
pointer is define by TOP attribute in the skeleton component. Initial value of
the TOP attribute also indicate size in bytes that a param fifo element must
have. Element of this link list are of struct TEvent type (Listing
\ref{structTevent}).

\begin{lstlisting}[caption=struct TEvent description, label=structTevent]
struct TEvent {
    struct TEvent* next; // pointer to next element or NULL if last.
    void* reaction;      // address of the method to execute
    void* THIS;          // component this.
    t_uword priority;    // priority of the component.
};
\end{lstlisting}

\subsubsection{Communication primitive api}
Using fifo primitive and coms and params fifo we implement high level primitive
which are then use by stub and skeleton component for communications :
\begin{itemize}
  \item AllocEvent() : Allocate a new element from a fifo to write message.
  \item PushEvent() : send a previously allocated element to a remote core.
  \item AcknowledgeEvent() : Acknoledge element reception so it be reuse for
  future communication.
  \item irqHandler() : each core implement an interrupt handler to receive
  message and then post a scheduler event (or a tasklet for CM driver). On the
  MMDSP side IT11() is the irq handler.
\end{itemize}

\subsection{How everything fit together}
Since we now have described all the basic element we will nows describe the flow
of call that occured. For that we will use api.call interface (Listing
\ref{apicall}) as an example.
\begin{lstlisting}[caption=api.call interface declaration, label=apicall]
interface api.call {
    void call(t_uint32 param);
}
\end{lstlisting}

\subsubsection{Host to MMDSP}
Host side sequence:
\begin{itemize}
  \item user call an interface using NMFCALL(itf, call)(1) which will call
  generated method st\_api\_call\_call(itf, 1) (Listing \ref{sthostgenerated}).
  \item st\_api\_call\_call() will marshal parameters on the stack and then call
  CM\_INTERNAL\_PushEventWithSize().
  \item CM\_INTERNAL\_PushEventWithSize() call CM\_OS\_PushEventWithSize() with
  the same exact parameters.
  \item According to target os, user call will end into kernel and will be split
  into CM\_ENGINE\_AllocEvent() to allocate an event into param fifo and then a
  call to CM\_ENGINE\_PushEvent() that will allocate an element from the com
  fifo and generate an interrupt to MMDSP.
\end{itemize}
MMDSP side sequence:
\begin{itemize}
  \item Interrupt is handle by IT11() function. This function will then perform
  the following actions:
  \begin{itemize}
    \item find the core responsible for the interrupt by calling
    sem\_GetFromCoreIdFromIrqSrc().
    \item Read element in com fifo associated with source core by calling
    fifo\_getNextElemToReadPointer().
    \item get the extendedField pointer which is located at offset
    EVENT\_ELEM\_EXTFIELD\_IDX inside fifo com element.
    \item get the head of event link list address which is located at offset
    EXTENDED\_FIELD\_BCTHIS\_OR\_TOP inside the extendedField table.
    \item Read the first element of the event link list address \verb#pEvent#
    (\verb#pEvent = *ppEventFifo;#).
    \item event link list cannot be empty since it has the same number of
    element as the corresponding param fifo. So if we detect an empty link list
    it means we got memory corruption and so we generate an internal panic.
    \item get the method index which is located at offset
    EVENT\_ELEM\_METHOD\_IDX inside fifo com element.
    \item set param address where marshal data are located of \verb#pEvent#
    located at offset EVENT\_ELEM\_PARAM\_IDX into com fifo.
    \item set method address to call which is located at offset
    EXTENDED\_FIELD\_BCDESC + idx into extendedField table.
    \item pop event from link list (\verb#(*ppEventFifo) = pEvent->event.next;#)
    \item mast all interrupts to avoid reentrency problem inside the scheduler
    and push event using Sched\_Event() call, then reenable interrupts.
  \end{itemize}
  \item later on method will be elected for running and ScheduleEvent() call
  with previously push event.
  \item skeleton method is call (Listing \ref{skmmdspgenerated}) and implement
  following steps:
  \begin{itemize}
    \item parameters located in param fifo are marshal.
    \item event is the push into original link list
    (\verb#_xyuv_event->event.next = ATTR(TOP);# and
    \verb#ATTR(TOP) = _xyuv_event;#)
    \item unmask all interrupts.
    \item call AcknowledgeEvent() to allow param fifo elem for new usage.
    \item call user method with correct parameters
    (\verb#target.call(call_param);#)
  \end{itemize}
\end{itemize}
\begin{lstlisting}[caption=st\_api\_call\_call() generated host code,
label=sthostgenerated]
static t_cm_error st_api_call_call(void* THIS, t_uint32 param) {
  t_cm_bf_host2mpc_handle host2mpcId = (t_cm_bf_host2mpc_handle )THIS;
  t_uint16 _xyuv_data[2];
    /* param <t_uint32> marshalling */
  _xyuv_data[0] = (t_uint16)param;
  _xyuv_data[0+1] = (t_uint16)(param >> 16);
  /* Event sending */
  return CM_INTERNAL_PushEventWithSize(host2mpcId, _xyuv_data, 2*2, 0);
}
\end{lstlisting}

\begin{lstlisting}[caption=MMDSP call() skeleton ,label=skmmdspgenerated]
void METH(call)(t_remote_event *_xyuv_event) {
  t_event_params_handle _xyuv_data = _xyuv_event->data;
  t_uint32 call_param;
    /* param <t_uint32> marshalling */
  call_param = (t_uint32)ext16to32((t_uint32)_xyuv_data[0+1], (t_uint32)_xyuv_data[0]);
  /* Server calling */
  _xyuv_event->event.next = ATTR(TOP);
  ATTR(TOP) = _xyuv_event;
  UNMASK_ALL_ITS();
  AcknowledgeEvent(ATTR(FIFO));
  nmfTraceActivity(TRACE_ACTIVITY_START, (t_uint24)target.THIS, (t_uint24)target.call);
  target.call(call_param);
  nmfTraceActivity(TRACE_ACTIVITY_END, (t_uint24)target.THIS, (t_uint24)target.call);
}
\end{lstlisting}

\subsubsection{MMDSP to host}
MMDSP side sequence:
\begin{itemize}
  \item user call an interface using NMFCALL(itf, call)(1) which will call
  generated method METH(call)(1) (Listing \ref{stmmdspgenerated}).
  \item stub call AllocEvent() that alloc an element from corresponding param
  fifo.
  \item paramaters are marshalled into param fifo element.
  \item stub call PushEvent().
  \begin{itemize}
    \item get an elem from the com fifo.
    \begin{itemize}
      \item fill it with method index.
      \item fill it with address of the param element.
      \item fill it with extented field address.
    \end{itemize}
    \item allow com element fifo usage and generate interrupt to host.
  \end{itemize}
\end{itemize}
HOST side sequence:
\begin{itemize}
  \item CM\_ProcessMpcEvent() CM driver will then handle this communication. See
  CM driver specification for more detail.
\end{itemize}

\begin{lstlisting}[caption=MMDSP call() stub ,label=stmmdspgenerated]
void METH(call)(t_uint32 param) {
  ENTER_CRITICAL_SECTION;
  {
    t_event_params_handle _xyuv_data = AllocEvent(ATTR(FIFO));
    /* param <t_uint32> marshalling */
    _xyuv_data[0] = (t_uint16)param;
    _xyuv_data[0+1] = (t_uint16)(param >> 16);
    /* Event sending */
    PushEvent(ATTR(FIFO), _xyuv_data, 0);
  }
  EXIT_CRITICAL_SECTION;
}
\end{lstlisting}

\section{Intra MMDSP communications}
In that case no param or com fifo are use but only a link list of event will be
use. In that case events will also carry marshal parameters.

So we will follow path of an asynchronous call.
\begin{itemize}
  \item user call an interface using NMFCALL(itf, call)(1) which will call
  generated method METH(call)(1) (Listing \ref{stasyncgenerated}).
  \item an event of struct EVENTapi\_call (Listing \ref{EVENTapicall}) is pop
  from event fifo list. This type is specific to an interface and define into
  generated code. It carry a struct TEvent and an union which is use for
  function parameters marshalling.
  \item we setup a reaction address to execute into event.
  \item function parameters are copy into dedicated space into event.
  \item event is push into scheduler by a Sched\_Event() call.
  \item later on event is elected by the scheduler for execution and a call to
  ScheduleEvent()id done.
  \item skeleton method is then call (Listing \ref{skasyncgenerated})
  \item function parameters are marshalled and put into the stack.
  \item event is push back into original fifo.
  \item user function is call with correct parameters.
\end{itemize}

\begin{lstlisting}[caption=MMDSP call() asynchronous stub
,label=stasyncgenerated]
void METH(call)(t_uint32 param) {
  ENTER_CRITICAL_SECTION; // OK, if only user emit event
  {
    struct EVENTapi_call* _xyuv_event = (struct EVENTapi_call*)ATTR(TOP);
    if(_xyuv_event == 0) Panic(EVENT_FIFO_OVERFLOW, 0);
    ATTR(TOP) = _xyuv_event->event.next;
    _xyuv_event->event.reaction = (void*)REACcall;
    _xyuv_event->data.call.param = param;
    nmfTraceActivity(TRACE_ACTIVITY_POST, (t_uint24)target.THIS, (t_uint24)target.call);
    Sched_Event(&_xyuv_event->event);
  }
  EXIT_CRITICAL_SECTION;
}
\end{lstlisting}

\begin{lstlisting}[caption=struct EVENTapi\_call, label=EVENTapicall]
struct EVENTapi_call {
  struct TEvent event;
  union {
    struct {
      t_uint32 param;
    } call;
  } data;
};
\end{lstlisting}

\begin{lstlisting}[caption=MMDSP call() asynchronous skeleton
,label=skasyncgenerated]
static void REACcall(struct EVENTapi_call* _xyuv_event) {
  // Declare parameters on stack or registers
  t_uint32 param;
  // Copy parameters
  param = _xyuv_event->data.call.param;
  _xyuv_event->event.next = ATTR(TOP);
  ATTR(TOP) = &_xyuv_event->event;
  UNMASK_ALL_ITS();
  nmfTraceActivity(TRACE_ACTIVITY_START, (t_uint24)target.THIS, (t_uint24)target.call);
  target.call(param);
  nmfTraceActivity(TRACE_ACTIVITY_END, (t_uint24)target.THIS, (t_uint24)target.call);
}
\end{lstlisting}
