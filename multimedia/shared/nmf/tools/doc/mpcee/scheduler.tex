%TODO : add a chapter on stack ?

\chapter{Hybrid Scheduler}
\section{Introduction}
Hybrid Scheduler implement three level of priorities named BACKGROUND, NORMAL
and URGENT and allow preemption between these priorities.
Implementation use one queue of task to execute by priority.
We will document how posting new reactions to these queues is done, how they are
schedule and execute. We will also describe preemption code.
\section{Globals data}
\textbf{FirstEvent[]} and \textbf{LastEvent[]} tables contains respectively head
and tail pointer for scheduler link list of event to execute. They contains 
pointers to struct TEvent elements (Listing \ref{struct TEvent}).
\begin{lstlisting}[caption=struct TEvent, label=struct TEvent]
struct TEvent {
  struct TEvent* next;
  void* reaction;
  void* THIS;
  t_uword priority;
};
\end{lstlisting}

\textbf{activePriority} contains the current running priority. When there is no
more reactions to execute then it return to value -1.

\textbf{switchOnGoing} is a boolean that indicates if a preemption is ongoing.
\section{Posting new reactions}
Posting a new reaction into scheduler queues is done using function
Sched\_Event(). This can occur either between host and MMDSP communication, MDSP
and MPDSP communication or during an asynchronous interface call.

It consist to insert a new struct TEvent into the correct schduler queue
according to priority and then perfom a preemption if active priority is lower
than posted event priority and no current preemption is ongoing.
\section{Scheduler loop}
Each time there is premption Schedule() function will be run again in a fresh
context. This function will call RawSchedule() from priority 2 down to previous
activePriority value.

RawSchedule() will execute event for a given priority until there are no more
event available and then return to caller.

\section{Preemption}
Preemption implementation use interrupt 2 of MMDSP (the lower level so that
others interrupts can still occur) to change current context by a new one that
will call Schedule(). Following steps describe in details how this occur. Code
that handle preemption is located in trampoline.c.

Preemption entry code begin in IT2() (Listing \ref{IT2}).
\begin{lstlisting}[caption=IT2(), label=IT2]
#pragma nosaveregs
_INTERRUPT void IT2(void) {
    pushrl2();
    deactivateIt();
    poprl2();
    switch_return();
}
\end{lstlisting}
Goal of IT2() is to jump back into trampoline() method instead of preempted
function. Buf before we must deactivate interrupts to avoid being interrupt in
context switch sequence since it's not reentrant. We then call switch\_return().

switch\_return() (Listing \ref{switchreturn}) is in charge to save current
return address in info1 global and then modify return address by trampoline()
address. So when we exit IT2() we start to execute trampoline().
\begin{lstlisting}[caption=switch\_return(), label=switchreturn]
asm void switch_return()
{
    stx_f       ax1, #_info2           // Memorise ax1
    /*
     * Change return address in system stack
     */
    ldxi_f      sp0, ax1                // read return address of IT2()
    stx_f       ax1, #_info1            // Memorise it
    ldx_f       #_trampolineptr, ax1    // Load new return address
    stxi_f      ax1, sp0                // Set new return address of IT2()
}
\end{lstlisting}

trampoline() will call trampoline\_0() where we will first change return address
so we return to save return address in info1 instead of returning to
trampoline(). We then call trampoline\_1() (Listing \ref{trampoline1}).
\begin{lstlisting}[caption=trampoline\_1(), label=trampoline1]
#pragma nosaveregs
void trampoline_1() {
    push_registers();
    Schedule();
    pop_registers();
}
\end{lstlisting}

trampoline\_1() will first save needed MMDSP registers (those not save by
callee) in push\_registers() before calling Schedule(). After Schedule() we will
restore MMDSP registers in pop\_registers() before returning to trampoline\_0()
and then into preempted code.
